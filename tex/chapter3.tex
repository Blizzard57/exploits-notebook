\setcounter{equation}{0}

\chapter{Routines}
\section{Introduction}
There should be some default set of processes to go through as 
soon as one sees a ctf question and knows the type of the question. 
These routines can help save some time, and in some cases, the 
process is so trivial that the process can be automated.\\\\ In 
this section, I am going to go through some of the processes 
followed by me and some codes that can help with the automation 
of the same.

\section{Cryptography}
For questions based on pure cryptography, each algorithm has its 
own attacks, its weaknesses and different tells. A computer itself 
cannot automate the complete process ( Atleast not me, currently) 
but after analysing the cipher given and the encryption given, one 
can break it easily by online tools or default algorithms.\\\\ 
These are the common cyphers for which there are attacks, but 
first, there is a need to analyse the problem, understand where 
the current scenario fits and apply the applicable algorithm.

\subsection{Substitution Cypher}
If the problem only gives an encrypted message without any key, 
first try these family of cyphers, as they are the most easy to 
break, and have a good change of success.

